\paragraph{huggingface/transformers.} Seleciona-se este repositorio por atender aos criterios de elegibilidade: (i) uso ativo do \textit{GitHub Actions}; (ii) execucao de testes automatizados nos \textit{pipelines}; (iii) atividade continua da comunidade; (iv) politica consistente de rotulagem de \textit{issues} de \textit{bug}. Esses fatores indicam maturidade operacional e disponibilidade de dados para a analise longitudinal proposta.
\paragraph{vercel/next.js.} Seleciona-se este repositorio por atender aos criterios de elegibilidade: (i) uso ativo do \textit{GitHub Actions}; (ii) execucao de testes automatizados nos \textit{pipelines}; (iii) atividade continua da comunidade; (iv) politica consistente de rotulagem de \textit{issues} de \textit{bug}. Esses fatores indicam maturidade operacional e disponibilidade de dados para a analise longitudinal proposta.
\paragraph{langflow-ai/langflow.} Seleciona-se este repositorio por atender aos criterios de elegibilidade: (i) uso ativo do \textit{GitHub Actions}; (ii) execucao de testes automatizados nos \textit{pipelines}; (iii) atividade continua da comunidade; (iv) politica consistente de rotulagem de \textit{issues} de \textit{bug}. Esses fatores indicam maturidade operacional e disponibilidade de dados para a analise longitudinal proposta.
\paragraph{pytorch/pytorch.} Seleciona-se este repositorio por atender aos criterios de elegibilidade: (i) uso ativo do \textit{GitHub Actions}; (ii) execucao de testes automatizados nos \textit{pipelines}; (iii) atividade continua da comunidade; (iv) politica consistente de rotulagem de \textit{issues} de \textit{bug}. Esses fatores indicam maturidade operacional e disponibilidade de dados para a analise longitudinal proposta.
\paragraph{apache/superset.} Seleciona-se este repositorio por atender aos criterios de elegibilidade: (i) uso ativo do \textit{GitHub Actions}; (ii) execucao de testes automatizados nos \textit{pipelines}; (iii) atividade continua da comunidade; (iv) politica consistente de rotulagem de \textit{issues} de \textit{bug}. Esses fatores indicam maturidade operacional e disponibilidade de dados para a analise longitudinal proposta.
\paragraph{expo/expo.} Seleciona-se este repositorio por atender aos criterios de elegibilidade: (i) uso ativo do \textit{GitHub Actions}; (ii) execucao de testes automatizados nos \textit{pipelines}; (iii) atividade continua da comunidade; (iv) politica consistente de rotulagem de \textit{issues} de \textit{bug}. Esses fatores indicam maturidade operacional e disponibilidade de dados para a analise longitudinal proposta.
\paragraph{metabase/metabase.} Seleciona-se este repositorio por atender aos criterios de elegibilidade: (i) uso ativo do \textit{GitHub Actions}; (ii) execucao de testes automatizados nos \textit{pipelines}; (iii) atividade continua da comunidade; (iv) politica consistente de rotulagem de \textit{issues} de \textit{bug}. Esses fatores indicam maturidade operacional e disponibilidade de dados para a analise longitudinal proposta.
\paragraph{n8n-io/n8n.} Seleciona-se este repositorio por atender aos criterios de elegibilidade: (i) uso ativo do \textit{GitHub Actions}; (ii) execucao de testes automatizados nos \textit{pipelines}; (iii) atividade continua da comunidade; (iv) politica consistente de rotulagem de \textit{issues} de \textit{bug}. Esses fatores indicam maturidade operacional e disponibilidade de dados para a analise longitudinal proposta.
\paragraph{grafana/grafana.} Seleciona-se este repositorio por atender aos criterios de elegibilidade: (i) uso ativo do \textit{GitHub Actions}; (ii) execucao de testes automatizados nos \textit{pipelines}; (iii) atividade continua da comunidade; (iv) politica consistente de rotulagem de \textit{issues} de \textit{bug}. Esses fatores indicam maturidade operacional e disponibilidade de dados para a analise longitudinal proposta.
\paragraph{apache/airflow.} Seleciona-se este repositorio por atender aos criterios de elegibilidade: (i) uso ativo do \textit{GitHub Actions}; (ii) execucao de testes automatizados nos \textit{pipelines}; (iii) atividade continua da comunidade; (iv) politica consistente de rotulagem de \textit{issues} de \textit{bug}. Esses fatores indicam maturidade operacional e disponibilidade de dados para a analise longitudinal proposta.
